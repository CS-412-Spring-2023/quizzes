\documentclass[addpoints,a4paper]{exam}

\usepackage{amsmath, amsfonts, amssymb, amsthm}
\usepackage{booktabs}
\usepackage{forest}
\usepackage{pythonhighlight}

\runningheader{CS 412 Algorithms}{Quiz 2A}{ID: \rule{.2\textwidth}{.5pt}}
\runningheadrule
\runningfootrule
\runningfooter{}{Page \thepage\ of \numpages}{}

% solution
\usepackage{draftwatermark}
\SetWatermarkText{Sample Solution}
\SetWatermarkLightness{.9}
\SetWatermarkScale{3}
\printanswers

\begin{document}
\begin{flushleft}
  { \large \textsf{\textbf{CS 412: Algorithms: Design and Analysis, L1, Quiz 2A, Spring 2023.}}}\vspace{.5em}
  
  \numquestions\ problems for \numpoints\ points on \numpages\ printed sides. Duration: 30 minutes. \today.
\end{flushleft}

Instructions:
\begin{enumerate}
  % \item Please observe the allowed time for this quiz as indicated on Canvas.
\item Enter your name and ID below and at the top of every side.
\item Solve the problems by hand in clear and legible handwriting in the provided space.
\item You may use the last side for rough work.
\item Provide precise and concise solutions.
\end{enumerate}

\noindent Student Name: \hrulefill \\[5pt]
\noindent Student ID: \hrulefill \\
\rule{\textwidth}{1pt}

\begin{questions}
  \question Consider the recurrence
    \begin{equation}
      \label{eq:rec1}
      T(n) = 4T(n-1) + 4T(n-2) + 4n, T(0) = 0, T(1) = 4.
    \end{equation}

  \begin{parts}
    \part[5] Design an algorithm whose time complexity is given by Equation \ref{eq:rec1}.
    \begin{solution}
\begin{python}
def f(n):
  # assume that the following takes 0 steps.
  if n == 0:
    return -1
  # assume that the following takes 4 steps.
  if n == 1:
    return 1+2+3+4
  # the general case follows.
  answer = f(n-1) + f(n-1) + f(n-1) + f(n-1) +
           f(n-2) + f(n-2) + f(n-2) + f(n-2)
  for i in 1 to 4*n:
    answer += i
  return answer
\end{python}
      \vspace{.4\textheight}
    \end{solution}
    \part[5] Justify that the time complexity of your algorithm is given by Equation \ref{eq:rec1}. Clearly state any relevant assumptions.
      \begin{solution}
        Assumptions for the base cases are mentioned in the comments in the algorithm. In the general case, the algorithm recurses 4 times with a problem size of $(n-1)$ and another 4 times with a problem size of $(n-2)$. Finally it iterates $4n$ times, performing a constant time operation in each iteration.

        Thus, the algorithm's time complexity is given by
        \[
          T(n) = 4T(n-1) + 4T(n-2) + 4n, T(0) = 0, T(1) = 4
        \]
        which is the same as Equation \ref{eq:rec1}.
      \vspace{.25\textheight}
    \end{solution}
  \end{parts}

\question Following is a linear non-homogeneous recurrence relation with constant coefficients. 
  \begin{equation}
    \label{eq:rec2}
    T(n) = 4T(n-1) - 4T(n-2) + 4n, T(0) = 0, T(1) = 4.
  \end{equation}
  Its solutions have the form
  \[
    T(n) = T(n)^{(p)} + T(n)^{(h)},
  \]
  where $T(n)^{(p)}$ is the \textit{particular solution} and $T(n)^{(h)}$ is the solution of the corresponding linear homogeneous recurrence relation. We are given that $T(n)^{(p)} = 4n+16$.
  \begin{parts}
    \part[5] Find a solution for $T(n)$.
      \begin{solution}
        The characteristic equation is $r^2 = 4r - 4$ which solves to $r=2,2$.

        Then, $T(n)^{(h)} = \alpha_12^n + \alpha_2n2^n$, and $T(n) = 4n + 16 + \alpha_12^n + \alpha_2n2^n$.

        Using the base cases,
        \begin{align*}
          T(0) & = 16 + \alpha_1 = 0 & \implies\alpha_1 = -16 & \\
          T(1) & = 4+16 -16\cdot2 + \alpha_2\cdot2 = 4 & \implies\alpha_2 = 8 & \\
        \end{align*}
        Therefore,
        \begin{align}
          T(n) & = 4n + 16 - 16\cdot2^n + 8n2^n \nonumber \\
               & = 4n + 16 - 2\cdot2^{n+3} + n2^{n+3} \nonumber\\
               & = 4(n + 4) + 2^{n+3}(n - 2). \label{eq:sol}
        \end{align}
    \end{solution}
    \newpage
    \part[5] Prove that your solution is correct.
      \begin{solution}
        We want to verify that\\
        Equation \ref{eq:sol}: $T(n) = 4(n + 4) + 2^{n+3}(n - 2)$.\\
        is a solution to\\
        Equation \ref{eq:rec2}: $T(n) = 4T(n-1) - 4T(n-2) + 4n$.

        We will do so by substituting Equation \ref{eq:sol} in Equation \ref{eq:rec2} and seeing if both sides are equal.
        \begin{proof}
          We substitute Equation \ref{eq:sol} in Equation \ref{eq:rec2}.
        \begin{align*}
          T(n) & = 4T(n-1) - 4T(n-2) + 4n\\
          4(n + 4) + 2^{n+3}(n - 2) & = 4\left(4(n+3) + 2^{n+2}(n-3)\right) - 4\left(4(n+2) + 2^{n+1}(n-4)\right) + 4n\\
          & = 4\left(4n + 12 + 2^{n+1}(2n-6) - 4n - 8 - 2^{n+1}(n-4) + n\right) \\
          & = 4\left(4 + 2^{n+1}(2n-6) - 2^{n+1}(n-4) + n\right) \\
          & = 4\left(n + 4 + 2^{n+1}(n-2)\right) \\
          & = 4(n + 4) + 2^{n+3}(n-2)
        \end{align*}
      \end{proof}
      \vspace{.6\textheight}
    \end{solution}
  \end{parts}
\end{questions}

\newpage
\centerline{\large Rough Work}

\end{document}

%%% Local Variables:
%%% mode: latex
%%% TeX-master: t
%%% End:
