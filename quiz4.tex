\documentclass[addpoints,a4paper]{exam}
\usepackage[a4paper]{geometry}

\usepackage{amsmath, amsfonts, amssymb, amsthm}
\usepackage{framed}
\usepackage{multirow}
\usepackage{tabularx}

\runningheader{CS 412 Algorithms}{Quiz 4: Greedy Algorithms}{ID: \rule{.2\textwidth}{.5pt}}
\runningheadrule
\runningfootrule
\runningfooter{}{Page \thepage\ of \numpages}{}

% solution
\usepackage{draftwatermark}
\SetWatermarkText{Sample Solution}
\SetWatermarkLightness{.9}
\SetWatermarkScale{3}
\printanswers

\begin{document}
\begin{flushleft}
  { \large \textsf{\textbf{CS 412: Algorithms: Design and Analysis, L1\&L4, Quiz 4, Spring 2023.}}}\vspace{.5em}
  
  \numquestions\ problems for \numpoints\ points on \numpages\ printed sides. Duration: 30 minutes. \today.
\end{flushleft}

Instructions:
\begin{enumerate}
  % \item Please observe the allowed time for this quiz as indicated on Canvas.
\item Enter your name and ID below and at the top of every side.
\item Solve the problems by hand in clear and legible handwriting in the provided space.
\item You may use the last side for rough work.
\item Provide precise and concise solutions.
\end{enumerate}

\noindent Student Name: \hrulefill \\[5pt]
\noindent Student ID: \hrulefill \\
\rule{\textwidth}{1pt}

\begin{questions}
\question  The \textit{coin change} problem is as follows.

  \begin{framed}
  \begin{quote}
    
    \underline{Input}:
    \begin{itemize}
    \item A non-empty set of positive coin denominations, $D=\{d_1, d_2, \ldots, d_n\}$.
    \item A positive integer $m$.
    \end{itemize}
    \underline{Output}: A sequence of non-negative integers $c_1, c_2, \ldots, c_n$ such that $\sum_{i=1}^n c_i d_i = m$ and that $\sum_{i=1}^n c_i$ is minimized.
  \end{quote}
\end{framed}
\begin{parts}
\part[1] Provide an example of $D$ and $m$ for which the greedy approach yields the correct solution. Also show the solution.
  \begin{solution}  Consider the following instance.
    \begin{itemize}
    \item $D=\{1, 5, 6, 9\}$.
    \item $m=9$.
    \end{itemize}
    The greedy approach yields the correct solution,  $(c_1, c_2, c_3, c_4)=(0,0,0,1)$, i.e. 1 coin. We know that it is correct because $9 = 9 \times 1$ and only 1 coin is used. Any other solution cannot use less than 1 coin.
  \end{solution}
\part[2] Provide an example of $D$ and $m$ for which the greedy approach does not yield the correct solution. Justify your answer.
  \begin{solution} Consider the following instance.
    \begin{itemize}
    \item $D=\{1, 5, 6, 9\}$.
    \item $m=11$.
    \end{itemize}
    The greedy approach yields the solution, $(c_1, c_2, c_3, c_4)=(q,0,0,1)$, i.e. a total value of 10. A solution with a larger total value is $(c_1, c_2, c_3, c_4)=(0,1,1,0)$, i.e. a total value of 11.
  \end{solution}
\end{parts}

  \newpage
\question  The \textit{knapsack} problem is as follows.
  
  \begin{framed}
  \begin{quote}
    
    \underline{Input}:
    \begin{itemize}
    \item Positive weights and values, $(w_1,v_1), (w_2,v_2), (w_3,v_3), \ldots, (w_n,v_n)$,  of a set of items, where $w_i$ and  $v_i$ are the weight and value respectively of item $i$.
    \item A positive value, $M$.
    \end{itemize}
    \underline{Output}: A set of quantities $c_1, c_2, \ldots, c_n$ such that $\sum_{i=1}^n c_i w_i \leq M$ and that $\sum_{i=1}^n c_i v_i$ is maximized.
  \end{quote}
\end{framed}
Two variants of this problem are:
  \begin{itemize}
  \item the \textit{0-1} knapsack problem: each $c_i$ can only be $0$ or $1$.
  \item the \textit{fractional} knapsack problem: each $c_i$ can be any real number between $0$ and $1$ inclusive.
  \end{itemize}

  \begin{parts}
  \part[3] Show that the greedy approach to the 0-1 knapsack problem does not yield the correct solution.
  \begin{solution}
    A greedy approach is to pick the most valuable item that fits in the knapsack. We show through a counterexample that this does not lead to the optimal solution.

    \begin{proof}
    Consider the following instance. $M=5$ and the items are as follows:
    \begin{center}
    \begin{tabular}{c|*4{c}}
      $w_i$ & $1$ & $2$ & $3$ & $4$ \\\hline
      $v_i$ & $1$ & $5$ & $6$ & $9$
  \end{tabular}
\end{center}
The greedy approach yields the solution, $(c_1, c_2, c_3, c_4)=(1,0,0,1)$, with a total value of 10. A solution with a higher value is $(c_1, c_2, c_3, c_4)=(0,1,1,0)$, which has a total value of 11.
\end{proof}
\end{solution}
\part[4] Devise a greedy approach for the fractional knapsack problem and show that the greedy choice at any stage is included in the eventual solution.
  \begin{solution}
    A greedy choice is to take as much as possible of item $m$ which has the highest value per unit weight, $\frac{v_m}{w_m} = \max_i \frac{v_i}{w_i}$. We prove by contradiction that the greedy choice is included in the optimal solution.
    \begin{proof}
      \begin{itemize}
      \item Consider a solution, $S$, that is optimal and includes an amount of item $m$ that is less than the maximum possible. That is, an unused amount of item $m$ remains.
      \item Construct a new solution, $S'$, from $S$ by replacing weight $x$ of any other item in the knapsack with weight $x$ of item $m$, where $x$ is chosen appropriately.
      \item $S'$ has a higher value than $S$. This contradicts our assumption that $S$ is optimal.
      \end{itemize}
  \end{proof}
  \end{solution}
  \end{parts}

\end{questions}

\newpage
\centerline{\large Rough Work}

\end{document}

%%% Local Variables:
%%% mode: latex
%%% TeX-master: t
%%% End:
